\section{Web Application}
%% (Reihenfolge noch vorläufig)
%% - Was ist Django?
%% - Wie setzt Django das MVC-Pattern um?
%% - Installation einschließlich Dependencies
%% - Benutzung der App (allgemein und Admin-Interface)
%% - Deployment
%% - Datenmodell (Ähnlichkeiten / Unterschiede zum XML-Schema)
%% - Zukünftige Erweiterungen (Templates und Views für Endnutzer,
%%   XML-Export)

\emph{Author: Tim Krones} \\

In addition to functionality for extracting and annotating the
contents of the Sbr-Regesten, the package also provides the basic
architecture for a web application. This chapter gives an overview of
the Django Framework for developing web application. It then describes
how to

\begin{itemize}
\item install necessary dependencies
\item run the application
\item use the admin interface to add or change data
\end{itemize}

and provides basic pointers for extending and deploying the
application. Also included is a detailed description of the data model
used for storing information extracted from the Sbr-Regesten in the
database associated with the application.

\subsection{The Django Web Framework}
\label{sec:django}

The information in this section is based on the official Django
documentation which can be found at
\url{https://docs.djangoproject.com/}.

Django is a Python Framework for rapid prototyping and development of
interactive web applications. It uses the MVC pattern to separate the
different tasks that are involved in creating interactive web
applications.

\subsubsection{The MVC Pattern}
\label{sec:mvc}

\href{https://en.wikipedia.org/wiki/Model_view_controller}{Model-View-Controller},
or MVC for short, is a
\href{https://en.wikipedia.org/wiki/Software_design_pattern}{Software Design Pattern}

commonly used by Web Frameworks such as Django and Ruby on Rails. The
basic idea of MVC is is to divide application logic into three layers.
The \emph{Model} layer is responsible for storing and operating on
data. This usually involves at least the basic
\href{https://en.wikipedia.org/wiki/CRUD}{CRUD} operations
\emph{Create}, \emph{Read}, \emph{Update}, and \emph{Delete}.The
\emph{View} layer takes care of presenting available data to end
users. \emph{Controllers} are responsible for handling user requests.
Depending on the type of request, this usually involves querying the
model layer for data, manipulating this data in various ways (if
necessary), and sending it off to the view layer for presentation.

Different frameworks interpret MVC in different ways; the next chapter
describes Django's implementation of this pattern.

\subsubsection{How Django Implements MVC}
\label{sec:django-mvc}

This chapter presents an overview of how Django interprets and
implements the MVC pattern. For an in-depth treatment of the
individual components, please consult the documentation at
\url{https://docs.djangoproject.com/}.

While Django's seperation of concerns is heavily influenced by the MVC
pattern conceptually, the framework uses a different terminology to
distinguish the individual components for dealing with (user)
requests, data, and presentation. The terminological differences tend
to confuse users that are new to Django or to working with MVC
frameworks in general, which makes it all the more important to
understand these differences before delving into Django development.

Django distinguishes between \emph{models}, \emph{templates}, and
\emph{views}, which is why the framework is commonly referred to as an
``MTV'' framework. The model layer in Django corresponds to the
concept of a model layer as it is defined (or at least commonly
understood) in the context of MVC. Django templates correspond to
views in MVC, and the responsibilities of Django views are similar to
those of controllers in MVC.

From an architectural point of view, a Django \emph{project} usually
consists of one or more Django \emph{apps}. Among other things, each
app includes a dedicated Python module for the model layer (called
\texttt{models.py}), two Python modules that are jointly responsible for
handling user requests (called \texttt{views.py} and \texttt{urls.py})
and a hierarchy of templates written in Django's template language.

The \texttt{models.py} module contains specialized Python classes
(called \emph{models}) which define the data model of a given Django
app. Each class corresponds to a table in the database of the project,
with additional tables being created as necessary to represent
relationships between different models.

The \texttt{views.py} module contains specialized Python functions
(called \emph{view functions}) for handling user requests. These
functions are responsible for querying the database for information,
manipulating that information if necessary, and rendering the
appropriate templates back to the user, filled with the information
that was requested. In this context, the \texttt{urls.py} module acts
as a kind of \emph{dispatcher}: It contains a mapping from URLs (or,
generally speaking, URL patterns) to appropriate view functions,
allowing Django to identify the actions it needs to take based on the
URL that was requested by the user.

\subsection{Installing and Using the Web Application}
\label{sec:webapp}

\subsection{Django Data Model}
\label{sec:data-model}
