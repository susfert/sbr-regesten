\section{Extraction Process: Overview}
\label{sec:overview}

The extraction process is based on an HTML version of the Sbr-Regesten. The latter was created using the \texttt{.doc} version of Sbr-regesten, obtained by the publishers. Microsoft Word 2010 was used to store it as \texttt{.html}. Compared to different open-source tools for convertion, it provided the best structural information for our purpose. The HTML file is stored in \texttt{sbr-regesten/html/sbr-regesten.hmtl}.\footnote{Due to minor inconsistencies in the HTML, the file was manually adapted at some places, see appendix xxx.}

In order to keep the architecture of the extraction process modular,
individual chapters of the Sbr-Regesten were extracted using separate
modules. Each of these modules works independently of the other
modules, but to get a well-formed, schema-conforming XML document,
they must be used in succession.

For ease of use, the extraction process was integrated into the
Sbr-Regesten Web Application as a \emph{management command}. This
chapter focuses on where to find the extraction modules and how they
are chained to form the overall extraction pipeline. For information
on (re-)running the extraction process, consult chapter \ref{sec:run}.

The order of extraction for the individual parts of the Sbr-Regesten
follows the order of chapters in the original text. The first module
is responsible for inserting the start tag of \emph{root element}
(\texttt{<sbr-regesten>}) into the XML document; the last module to
process the HTML version of the Sbr-Regesten inserts the corresponding
end tag (\texttt{</sbr-regesten}). Book chapters and the modules
responsible for extracting them are listed in the following table:

\begin{figure}[h]
  \centering
  \begin{tabular}{l|l}
    \hline
    Chapter & Extraction Module \\
    \hline
    Frontmatter & \texttt{frontmatter\_extractor.py} \\
    Table of Contents & \texttt{toc\_extractor.py} \\
    Preface & \texttt{preface\_extractor.py} \\
    Bibliography & \texttt{bibliography\_extractor.py} \\
    List of Abbreviations & \texttt{abbrev\_extractor.py} \\
    List of Initials & \texttt{initials\_extractor.py} \\
    Regests & \texttt{regest\_extractor.py} \\
    List of Archives & \texttt{archives\_extractor.py} \\
    Index & \texttt{index\_extractor.py} \\
    \hline
  \end{tabular}
  \caption{Book chapters and corresponding extraction modules}
  \label{fig:extraction-modules}
\end{figure}

The extraction modules are located in the
\texttt{sbr-regesten/extraction} directory. Each one of them uses the
HTML version of the Sbr-Regesten stored in
\texttt{sbr-regesten/html/sbr-regesten.html} as input and implements a
function called \texttt{extract\_<part-of-book>}. This function takes
no arguments and serves as an entry point for extracting the chapter
it is responsible for. The \texttt{extract.py} module located in

\begin{verbatim}
sbr-regesten/regesten_webapp/management/commands
\end{verbatim}

chains the calls to the entry points in the appropriate order:

\begin{verbatim}
# Imports ...

class Command(NoArgsCommand):
    help = 'Starts and directs the extraction process for the Sbr-Regesten'

    def handle_noargs(self, **options):
        frontmatter_extractor.extract_frontmatter()
        toc_extractor.extract_toc()
        preface_extractor.extract_preface()
        bibliography_extractor.extract_bibliography()
        abbrev_extractor.extract_abbrevs()
        initials_extractor.extract_initials()
        regest_extractor.extract_regests()
        archives_extractor.extract_archives()
        index_extractor.extract_index()
\end{verbatim}

The following chapters describe in detail how individual parts of the
Sbr-Regesten are extracted, and also provide information about
relevant parts of the XML schema.
